\documentclass{article}\usepackage[]{graphicx}\usepackage[]{color}
%% maxwidth is the original width if it is less than linewidth
%% otherwise use linewidth (to make sure the graphics do not exceed the margin)
\makeatletter
\def\maxwidth{ %
  \ifdim\Gin@nat@width>\linewidth
    \linewidth
  \else
    \Gin@nat@width
  \fi
}
\makeatother

\definecolor{fgcolor}{rgb}{0.345, 0.345, 0.345}
\newcommand{\hlnum}[1]{\textcolor[rgb]{0.686,0.059,0.569}{#1}}%
\newcommand{\hlstr}[1]{\textcolor[rgb]{0.192,0.494,0.8}{#1}}%
\newcommand{\hlcom}[1]{\textcolor[rgb]{0.678,0.584,0.686}{\textit{#1}}}%
\newcommand{\hlopt}[1]{\textcolor[rgb]{0,0,0}{#1}}%
\newcommand{\hlstd}[1]{\textcolor[rgb]{0.345,0.345,0.345}{#1}}%
\newcommand{\hlkwa}[1]{\textcolor[rgb]{0.161,0.373,0.58}{\textbf{#1}}}%
\newcommand{\hlkwb}[1]{\textcolor[rgb]{0.69,0.353,0.396}{#1}}%
\newcommand{\hlkwc}[1]{\textcolor[rgb]{0.333,0.667,0.333}{#1}}%
\newcommand{\hlkwd}[1]{\textcolor[rgb]{0.737,0.353,0.396}{\textbf{#1}}}%
\let\hlipl\hlkwb

\usepackage{framed}
\makeatletter
\newenvironment{kframe}{%
 \def\at@end@of@kframe{}%
 \ifinner\ifhmode%
  \def\at@end@of@kframe{\end{minipage}}%
  \begin{minipage}{\columnwidth}%
 \fi\fi%
 \def\FrameCommand##1{\hskip\@totalleftmargin \hskip-\fboxsep
 \colorbox{shadecolor}{##1}\hskip-\fboxsep
     % There is no \\@totalrightmargin, so:
     \hskip-\linewidth \hskip-\@totalleftmargin \hskip\columnwidth}%
 \MakeFramed {\advance\hsize-\width
   \@totalleftmargin\z@ \linewidth\hsize
   \@setminipage}}%
 {\par\unskip\endMakeFramed%
 \at@end@of@kframe}
\makeatother

\definecolor{shadecolor}{rgb}{.97, .97, .97}
\definecolor{messagecolor}{rgb}{0, 0, 0}
\definecolor{warningcolor}{rgb}{1, 0, 1}
\definecolor{errorcolor}{rgb}{1, 0, 0}
\newenvironment{knitrout}{}{} % an empty environment to be redefined in TeX

\usepackage{alltt}
\usepackage{Sweave}
\usepackage{float}
\usepackage{graphicx}
\usepackage{tabularx}
\usepackage{siunitx}
\usepackage{geometry}
\usepackage{pdflscape}
\usepackage{mdframed}
\usepackage{natbib}
\bibliographystyle{..//bib/styles/besjournals.bst}
\usepackage[small]{caption}
\setlength{\captionmargin}{30pt}
\setlength{\abovecaptionskip}{0pt}
\setlength{\belowcaptionskip}{10pt}
\topmargin -1.5cm        
\oddsidemargin -0.04cm   
\evensidemargin -0.04cm
\textwidth 16.59cm
\textheight 21.94cm 
%\pagestyle{empty} %comment if want page numbers
\parskip 7.2pt
\renewcommand{\baselinestretch}{1.5}
\parindent 0pt
\usepackage{lineno}
\linenumbers

\newmdenv[
  topline=true,
  bottomline=true,
  skipabove=\topsep,
  skipbelow=\topsep
]{siderules}

%% R Script


\IfFileExists{upquote.sty}{\usepackage{upquote}}{}
\begin{document}
\noindent \textbf{\Large{Regional Risk Outline}}

\noindent Authors:\\
C. J. Chamberlain $^{1,2}$, B. I. Cook $^{3}$, I. Morales Castilla $^{1,4}$ \& E. M. Wolkovich $^{1,2}$
\vspace{2ex}\\
\emph{Author affiliations:}\\
$^{1}$Arnold Arboretum of Harvard University, 1300 Centre Street, Boston, Massachusetts, USA; \\
$^{2}$Organismic \& Evolutionary Biology, Harvard University, 26 Oxford Street, Cambridge, Massachusetts, USA; \\
$^{3}$NASA Goddard Institute for Space Studies, New York, New York, USA; \\
$^{4}$Edificio Ciencias, Campus Universitario 28805 Alcalá de Henares, Madrid, Spain \\
\vspace{2ex}
$^*$Corresponding author: 248.953.0189; cchamberlain@g.harvard.edu\\

\renewcommand{\thetable}{\arabic{table}}
\renewcommand{\thefigure}{\arabic{figure}}
\renewcommand{\labelitemi}{$-$}
\setkeys{Gin}{width=0.8\textwidth}

%%%%%%%%%%%%%%%%%%%%%%%%%%%%%%%%%%%%%%%%%%%%%%%
%%%%%%%%%%%%%%%%%%%%%%%%%%%%%%%%%%%%%%%%%%%%%%%

%%%%% 1) PIECE TOGETHER THE PUZZLE! %%%%%
%%%%% 2) FORMULATE QUESTIONS %%%%%
%%%%% 3) ONCE THE INTRO IS SOLID, THEN MOVE ON TO METHODS! %%%%%%


\section*{Introduction}
\begin{enumerate}
\item Temperate tree and shrub species are at risk of damage from late spring freezing events, also known as false springs.
\begin{enumerate}
\item However, the extent of damage and the frequency and intensity of false spring events is still largely unknown.
\item Individuals that initiate budburst before the last spring freeze are at risk of leaf tissue loss, damage to the xylem, and slowed canopy development \citep{Gu2008, Hufkens2012}.
\item Temperate plants are exposed to freezing temperatures numerous times throughout the year, however, individuals are most at risk to damage from stochastic spring frosts, when frost tolerance is lowest \citep{Sakai1987}.
\item The growing season is lengthening across many regions in the northern hemisphere \citep{Chen2005, Liu2006, Kukal2018}, but last spring frosts still pose a threat in many of these regions \citep{Wypych2016a}.
\item False spring events can result in photosynthetic tissue loss, which could potentially impact multiple years of growth and, with the growing season extending, individuals could be exposed to more frosts in the future \citep{Liu2018}.
\item For these reasons, episodic frosts are one of the largest limiting factors in species range limits \citep{Kollas2014}. 
\end{enumerate}


\item Plant phenology, which is defined as the timing of recurring life-history events such as budburst, strongly tracks shifts in climate \citep{Cleland2007, Wolkovich2012}.
\begin{enumerate}
\item Trees and shrubs in temperate regions optimize growth by using three cues to initiate budburst: low winter temperatures, warm spring temperatures, and increasing spring daylengths.
\item With climate change advancing, this interaction of cues may shift spring phenologies both across and within species. 
\item Due to the changing climate, spring onset is advancing and many temperate tree and shrub species are initiating leafout 4-6 days earlier per $^{\circ}$C of warming \citep{Wolkovich2012, IPCC2014}.
\item However, last spring freeze dates are not predicted to advance at the same rate as spring onset in some regions of the world \citep{Inouye2008, Martin2010, Labe2016, Sgubin2018}, potentially amplifying the effects of false spring events in these regions.
\item Major false spring events have been recorded in recent years and have found it can take 16-38 days for trees to refoliate \citep{Gu2008, Augspurger2009, Augspurger2013, Menzel2015}, which can detrimentally affect crucial processes such as carbon updake and nutrient cycling \citep{Hufkens2012, Richardson2013, Klosterman2018}.
\end{enumerate}


\item Temperate plants have evolved to minimize false spring damage through a myriad of strategies, with the most effective being avoidance: plants must exhibit flexible spring phenologies in order to maximize growth and minimize frost risk by timing budburst effectively \citep{Polgar2011, Basler2014}.
\begin{enumerate}
\item Plants growing in forest systems tend to exhibit staggered days of budburst.
\item Lower canopy species typically initiate budburst earlier in the season in order to utilize available resources such as light, whereas larger canopy species usually initiate budburst later in the season.
\item Thus, there is a trade-off between growing season length and frost risk. 
\item Frost tolerance greatly diminishes once individuals exit the dormancy phase (i.e. processes leading to budburst) through full leaf expansion \citep{Vitasse2014, Lenz2016}.
\item Individuals that initiate budburst earlier in the season are more frost resistant \citep{Korner2016}, however, as climate change advances, less frost resistant individuals may start initiating budburst before the last freeze date.
\end{enumerate}

\item Many studies have assessed the interplay between cue interactions and budburst dates by investigating potenital latitudinal effects \citep{Partanen2004, Viheraaarnio2006, Caffarra2011, Zohner2016, Gauzere2017}.
\begin{enumerate}
\item However, recent studies have demonstrated regional effects may be more closely related to false spring risk: whether via altitudinal variation \citep{Vitra2017} or distance from the coast \citep{Wypych2016a}.
\item By better understanding these regional climatic implications, we may be able to determine which regions may be at risk currently and which regions may become more at risk in the future.
\end{enumerate}
%\item Individuals from more Northern provenances tend to be more susceptible to spring frost damage, whereas more Southern individuals are more sensitive to fall frosts \citep{Montwe2018}.

\item There is large debate over whether or not spring freeze damage will increase \citep{Hannenin1991, Augspurger2013, Labe2016}, remain the same \citep{Scheifinger2003} or even decrease \citep{Kramer1994} with climate change.
\begin{enumerate}
\item Some research suggests false spring incidence has declined in many regions (i.e. across parts of North America and Asia), however the prevalence of spring frosts has consistently increased across Europe since 1982 \citep{Liu2018}.
\item The North Atlantic Ossilation (NAO) index is often used to describe winter and spring circulation across Europe. 
\item More positive NAO phases tend to result in higher than average winter and early spring temperatures, and with climate change, higher NAO phases has correlated to even earlier budburst dates \citep{Chmielewski2001}, however it is unclear if more positive NAO phases also translates to more false springs.
\item By improving and identifying budburst and climate trends in recent years, we could potentially amplify our predictability of future projections in false springs.
\item A false spring, in this study, was recorded when temperatures fell below -2$^{\circ}$ \citep{Schwartz1993} between budburst and leafout (CITE Rethinking here?)
\item For this purpose, we assessed the number of false springs that occured across 11,684 sites around Europe, spanning altitudinal and coastal gradients, using observed phenological data (857,004 observations) for six temperate, deciduous trees and combined that with daily gridded climate data for each site that extended from 1950-2016. %% Do we want a table here with a breakdown of sites, species, and years? Probably in the supplement? 
\item Since the primary aim is to predict false spring incidence in a changing climate, we split our data at 1983 to capture reported temporal shifts in temperature trends \citep{Stocker2013, Kharouba2018}.
\item We predicted that: (1) Budburst timing would be earlier for all six species after the early to mid 1980s, (2) species that initiate budburst earlier would experience more false springs, and (3) there would be different regional effects on false spring incidence, especially coupled with climate change.
\end{enumerate}
\end{enumerate}
%% WHERE DOES THIS GO? \item By using only gridded climate data from March 1st until June 30th, we compared the first half of our data (1950-1983) to the latter half (1984-2016) and found that some regions experienced increased exposure to false spring conditions, whereas many regions experienced fewer or similar conditions (Figure \ref{fig:region}). %% supplement?

\section*{Methods}
\subsection*{Phenological Data}
\begin{enumerate}
\item PEP725
\item Species are \textit{Aesculus hippocastanum L.}, \textit{Alnus glutinosa} (L.) Gaertn., \textit{Betula pendula} Roth., \textit{Fagus sylvatica} Ehrh., \textit{Fraxinus excelsior} L., \textit{Quercus robur} L.
\item Narrowed down to 6 species based on the following criteria:
\begin{itemize}
\item Needed to be temperate, deciduous species that were not cultivars or used for crops
\item Needed at least 100,000 useable observations per species - useable being BBCH 11
\item Needed to span over half of all the sites used in the study
\item Needed observations for every year in the study
\end{itemize}
\item Used BBCH 11 as leafout
\item Then subtracted 12 days to determine budburst and subsequently the timeframe we will use to establish false spring conditions. \citep{Donnelly2017}
\end{enumerate}
\subsection*{Climate Data}
\begin{enumerate}
\item Used data from E-OBS
\item Daily gridded climate data, daily minimum temperature was used for false spring (-2$^{\circ}$)
\item Mean spring temperature used average daily temperature from February 1st through April 30th \citep{Korner2016} - use only the spring temperature to capture spring warming, rather than entire year which has not yet been linked to any biological process.
\item Split at 1983 - to capture climate change effects and was an even split
\end{enumerate}
\subsection*{Data Analysis}
\begin{enumerate}
\item Parameters for a false spring: temperatures had to fall below -2$^{\circ}$ at least once between budburst and leafout.
\item Bayesian hierarchical model
\item BRMS in R 
\item Poisson distribution
\item Do I need to mention Odyssey??
\item Scaled elevation parameter to be consistent with other parameters in the model
\item Only random slopes with species - otherwise too big
\item Adjusted for spatial autocorrelation using method proposed by \citep{Baumen2017} and then regressed eigenvectors on the number of false springs to establish a spatial parameter. % get citation from Nacho regarding regression
\end{enumerate}



\section*{Results}


\section*{Discussion \& Conclusion}


\bibliography{..//bib/regionalrisk.bib}

\section*{Tables and Figures}

{\begin{figure} [H]
  -\begin{center}
  -\includegraphics[width=12cm]{..//figures/FS_Diff.pdf}
  -\caption{Number of years with freezing events that occured before temperature shifts related to climate change began (1951-1983) as compared to after reported climate shifts (1984-2016). If temperatures fell below -2$^{\circ}$C between March 1 and June 30, a year with a spring freeze was tallied. Some regions experienced more years with spring freezes after climate change began, whereas other years experienced the same number or even fewer years with spring freezes. Regions that had more years with spring freezes after climate change began are blue and green and regions that had fewer freezes are depicted in red.}\label{fig:region}
  -\end{center}
  -\end{figure}}
  
{\begin{figure} [H]
  -\begin{center}
  -\includegraphics[width=14cm]{..//figures/BB_gis.pdf}
  -\caption{The average day of budburst is mapped by site for each species. Species are ordered by day of budburst starting with \textit{Betula pendula} as the earliest budburst date to \textit{Fraxinus excelsior}. Earlier budburst dates are blue and later budburst dates are in red. }\label{fig:bbmap}
  -\end{center}
  -\end{figure}}
  
{\begin{figure} [H]
  -\begin{center}
  -\includegraphics[width=14cm]{..//figures/PropSitesbyYrwBB.png}
  -\caption{The black line indicates the proportion of sites that had false spring conditions for each year across all species. The blue line is a smoothing spline, indicating the trend of average day of budburst for each year for each species. Species are ordered by average day of budburst, with the earliest being \textit{Betula pendula} and the latest being \textit{Fraxinus excelsior}.}\label{fig:fsprop}
  -\end{center}
  -\end{figure}}
  
%{\begin{figure} [H]
%  -\begin{center}
%  -\includegraphics[width=12cm]{..//figures/space_mapped.pdf}
%  -\caption{Space Parameter}\label{fig:space}
%  -\end{center}
%  -\end{figure}}


\end{document}
