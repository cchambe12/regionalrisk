\Sconcordance{concordance:RegionalRisk.tex:RegionalRisk.Rnw:%
1 38 1 50 0 1 6 173 1 1 0}
